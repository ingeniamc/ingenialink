\href{https://travis-ci.org/ingeniamc/ingenialink}{\tt } \href{https://ci.appveyor.com/project/gmarull/ingenialink}{\tt }

{\ttfamily libingenialink} is a portable, pure C implementation library for simple motion control tasks and communications with Ingenia drives.

\href{http://www.ingeniamc.com}{\tt }

\subsection*{What it can do}

The library provides\+:


\begin{DoxyItemize}
\item Simple motion control functions (homing, profile position, etc.)
\item Communications A\+PI for Ingenia drives (multiple protocols supported)
\item Load and use Ingenia\+Dictionary X\+ML dictionaries
\item Operate directly using units (e.\+g. degrees, rad/s, etc.)
\item Register polling and monitoring for scope applications
\item Servo listing and monitoring
\item Object oriented interface
\item Thread-\/safe communications
\item Descriptive and detailed error messages
\end{DoxyItemize}

\subsection*{Building libingenialink}

{\ttfamily libingenialink} depends on \href{https://github.com/ingeniamc/sercomm}{\tt libsercomm} and \href{https://xmlsoft.org}{\tt libxml2}. A couple of sections below you will find some instructions on how to build and install them. {\ttfamily libingenialink} can be built and installed on any system like this\+:


\begin{DoxyCode}
1 cmake -H. -B\_build -DCMAKE\_INSTALL\_PREFIX=$INSTALL
2 cmake --build \_build
3 cmake --build \_build --target install
\end{DoxyCode}


Note that a {\ttfamily I\+N\+S\+T\+A\+LL} is the installation folder.

\subsubsection*{Build options}

The following build options are available\+:


\begin{DoxyItemize}
\item {\ttfamily W\+I\+T\+H\+\_\+\+P\+R\+O\+T\+\_\+\+E\+U\+SB} (ON)\+: Build {\ttfamily E\+U\+SB} protocol support.
\item {\ttfamily W\+I\+T\+H\+\_\+\+P\+R\+O\+T\+\_\+\+M\+CB} (O\+FF)\+: Build {\ttfamily M\+CB} protocol support (E\+X\+P\+E\+R\+I\+M\+E\+N\+T\+AL).
\item {\ttfamily W\+I\+T\+H\+\_\+\+E\+X\+A\+M\+P\+L\+ES} (O\+FF)\+: When enabled, the library usage example applications will be built.
\item {\ttfamily W\+I\+T\+H\+\_\+\+D\+O\+CS} (O\+FF)\+: When enabled the A\+PI documentation can be built.
\item {\ttfamily W\+I\+T\+H\+\_\+\+P\+IC} (O\+FF)\+: When enabled, generated code will be position independent. This may be useful if you want to embed ingenialink into a dynamic library.
\end{DoxyItemize}

Furthermore, {\itshape standard} C\+Make build options can be used. You may find useful to read this list of \href{https://cmake.org/Wiki/CMake_Useful_Variables}{\tt useful C\+Make variables}.

\subsection*{Dependencies}

As mentioned before, {\ttfamily libingenialink} depends on \href{https://github.com/ingeniamc/sercomm}{\tt libsercomm} and \href{https://xmlsoft.org}{\tt libxml2}, both referenced in the \href{https://github.com/ingeniamc/ingenialink/tree/master/external}{\tt external} folder as submodules. Therefore, if building them make sure to initialize the submodules first\+:


\begin{DoxyCode}
1 git submodule update --init --recursive
\end{DoxyCode}


Below you can find some building instructions for dependencies. Note that {\ttfamily I\+N\+S\+T\+A\+LL} is the installation folder.

\subsubsection*{libsercomm}

{\ttfamily libsercomm} also uses C\+Make, so it can be built and installed on any system like this\+:


\begin{DoxyCode}
1 cd external/sercomm
2 cmake -H. -B\_build -DCMAKE\_INSTALL\_PREFIX=$INSTALL
3 cmake --build \_build --target install
\end{DoxyCode}


\subsubsection*{libxml2}

Athough {\ttfamily libxml2} is multiplatform, the building process can be somewhat painful on some systems, specially on Windows. This is why we provide a C\+Make script to build it on the systems we support. It can be built and installed like this\+:


\begin{DoxyCode}
1 cd external/libxml2
2 cmake -H. -B\_build -DCMAKE\_INSTALL\_PREFIX=$INSTALL
3 cmake --build \_build --target install
\end{DoxyCode}


If using Linux, we actually recommend installing the library packages from the official repositories. For example in Debian/\+Ubuntu systems\+:


\begin{DoxyCode}
1 sudo apt install libxml2-dev
\end{DoxyCode}


On recent versions of mac\+OS, it seems to be already installed on the system. If not, you can also use \href{https://brew.sh}{\tt brew} to install it.

\subsection*{Coding standards}

{\ttfamily libingenialink} is written in \href{http://en.wikipedia.org/wiki/ANSI_C}{\tt A\+N\+SI C} (C99), so any modern compiler should work.

Code is written following the \href{https://www.kernel.org/doc/html/latest/process/coding-style.html}{\tt Linux Kernel coding style}. You can check for errors or suggestions like this (uses {\ttfamily checkpatch.\+pl})\+:


\begin{DoxyCode}
1 cmake --build build --target style\_check
\end{DoxyCode}
 